\setcounter{chapter}{0}
\chapter{Literature review}
\label{chap:stateoftheart}
This chapter discusses previous research about the topic, providing a brief 
introduction to the approach we adopted.
%, evaluating, in particular, their shortcomings (difetti). 

\glsreset{ai}
\glsreset{dl}
\glsreset{rl} 
\glsreset{dnn}

\bigskip
Siccome non è mai esistito un sistema che riconosce la mano davvero efficiente e in tmepo reale allora non è mai esistito la possibilità di poter controllare le traiettorie con la mano. In seguito alla soluzione recente fatta dai ricercatori di google che ha risolto i diversi problemi di performance e precisione nel riconoscere la mano, ora è possibile ragionare su come eseguire delle traiettorie con la mano. QUesto mio studio ha l'obiettivo quindi di aggiungere un'ulteriore mattoncino al lavoro svolto dai ragazzi di google per controllare il drone usando soltanto la mano, risolvendo quindi diversi problemi.
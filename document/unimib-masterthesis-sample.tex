\documentclass[mscthesis, 11pt, oneside, openany]{usiinfthesis}

\usepackage[utf8]{inputenc} 
\usepackage[T1]{fontenc}
\usepackage{textcomp}
\usepackage{amsmath} % new

\usepackage{setspace}

\usepackage[font=small,labelfont=bf]{caption}
\usepackage{float}
\usepackage{blindtext}

\setcounter{tocdepth}{2}
\setcounter{secnumdepth}{6}

\usepackage{tocloft}
\setlength{\cftfignumwidth}{3em}

\usepackage{enumitem}

\usepackage{graphicx}
\pdfsuppresswarningpagegroup=1
\usepackage{caption}
%\usepackage[justification=justified,singlelinecheck=false]{caption}
\usepackage{subcaption}

\usepackage{url}
\makeatletter
\g@addto@macro{\UrlBreaks}{\UrlOrds}
\makeatother
\PassOptionsToPackage{hyphens}{url}
%Import the natbib package and sets a bibliography  and citation styles
%\bibliographystyle{dinat}
%\bibliographystyle{alpha}
%\bibliographystyle{dcu}
\bibliographystyle{plainnat}
\setcitestyle{authoryear,open={[},close={]}}

\newfloat{Equation}{htbp}{equ}[chapter]
\newcommand{\listequationsname}{Equations}

\usepackage{xcolor}
\usepackage{listings,lstautogobble}
\definecolor{gray}{gray}{0.5}
\colorlet{commentcolour}{green!50!black}
\colorlet{stringcolour}{red!60!black}
\colorlet{keywordcolour}{blue}
\colorlet{exceptioncolour}{yellow!50!red}
\colorlet{commandcolour}{magenta!90!black}
\colorlet{numpycolour}{blue!60!green}
\colorlet{literatecolour}{magenta!90!black}
\colorlet{promptcolour}{green!50!black}
\colorlet{specmethodcolour}{violet}
\newcommand*{\literatecolour}{\textcolor{literatecolour}}
\newcommand*{\pythonprompt}{\textcolor{promptcolour}{{>}{>}{>}}}
\lstdefinestyle{python}{
	language=python,
	showtabs=true,
	tab=,
	tabsize=4,
	basicstyle=\ttfamily\footnotesize,
	stringstyle=\color{stringcolour},
	showstringspaces=false,
	keywordstyle=\color{keywordcolour}\bfseries,
	emph={as,and,break,class,continue,def,yield,del,elif ,else,%
		except,exec,finally,for,from,global,if,in,%
		lambda,not,or,pass,print,raise,return,try,while,assert,with},
	emphstyle=\color{blue}\bfseries,
	emph={[2]True, False, None},
	emphstyle=[3]\color{commandcolour},
	morecomment=[s]{"""}{"""},
	commentstyle=\color{commentcolour}\slshape,
	emph={array, matmul, ones, transpose, float32},
	emphstyle=[4]\color{numpycolour},
	emph={[5]assert,yield},
	emphstyle=[5]\color{keywordcolour}\bfseries,
	emph={[6]range},
	emphstyle={[6]\color{keywordcolour}\bfseries},
	literate=*%
	{:}{{\literatecolour:}}{1}%
	{=}{{\literatecolour=}}{1}%
	{-}{{\literatecolour-}}{1}%
	{+}{{\literatecolour+}}{1}%
	{*}{{\literatecolour*}}{1}%
	{**}{{\literatecolour{**}}}2%
	{/}{{\literatecolour/}}{1}%
	{//}{{\literatecolour{//}}}2%
	{!}{{\literatecolour!}}{1}%
	{<}{{\literatecolour<}}{1}%
	{>}{{\literatecolour>}}{1}%
	{>>>}{\pythonprompt}{3},
	frame=trbl,
	rulecolor=\color{black!40},
	backgroundcolor=\color{gray!5},
	breakindent=.5\textwidth,
	frame=single,
	breaklines=true,
	basicstyle=\ttfamily\footnotesize,%
	keywordstyle=\color{keywordcolour},%
	emphstyle={[7]\color{keywordcolour}},%
	emphstyle=\color{exceptioncolour},%
	literate=*%
	{:}{{\literatecolour:}}{2}%
	{=}{{\literatecolour=}}{2}%
	{-}{{\literatecolour-}}{2}%
	{+}{{\literatecolour+}}{2}%
	{*}{{\literatecolour*}}2%
	{**}{{\literatecolour{**}}}3%
	{/}{{\literatecolour/}}{2}%
	{//}{{\literatecolour{//}}}{2}%
	{!}{{\literatecolour!}}{2}%
	{<}{{\literatecolour<}}{2}%
	{<=}{{\literatecolour{<=}}}3%
	{>}{{\literatecolour>}}{2}%
	{>=}{{\literatecolour{>=}}}3%
	{==}{{\literatecolour{==}}}3%
	{!=}{{\literatecolour{!=}}}3%
	{+=}{{\literatecolour{+=}}}3%
	{-=}{{\literatecolour{-=}}}3%
	{*=}{{\literatecolour{*=}}}3%
	{/=}{{\literatecolour{/=}}}3%
}
\lstnewenvironment{python}
{\lstset{style=python}}
{}

% Load the package with the acronym option
\usepackage[titletoc,page]{appendix}
\makeatletter\@openrightfalse\makeatother

\usepackage[acronym,automake,nonumberlist]{glossaries}
%\usepackage[noindex]{glossaries-extra}

% Generate the glossary
\makeglossaries
\renewcommand*\glspostdescription{\dotfill}
\renewcommand*\acronymname{ }

\newcommand{\nomNoPrint}[3]{\newglossaryentry{#1}{
		name={#2},
		symbol={#2},
		description={#3},
		sort={A\three@digits{\value{page}}}
	}\glsadd[format=hyperbf]{#1}}   
\makeatother

\newcommand{\nom}[3]{\nomNoPrint{#1}{#2}{#3}#2}

\newglossaryentry{fig}{
	name={Fig.},
	description={Figure}
}


\newglossaryentry{tab}{
	name={Tab.},
	description={Table}
}

\newglossaryentry{cm}{
	name={cm},
	description={centimetre}
}

\newglossaryentry{s}{
	name={s},
	description={seconds}
}


\newglossaryentry{h}{
	name={h},
	description={hours}
}

\newglossaryentry{cm/s}{
	name={cm/s},
	description={centimeters per second}
}

\newglossaryentry{wrt}{name={w.r.t.},description={with respect to}}

\newglossaryentry{Hz}{name={Hz},description={hertz}}

\newacronym{ai}{AI}{Artificial Intelligence}
\newacronym{ml}{ML}{Machine Learning}
\newacronym{dl}{DL}{Deep Learning}
\newacronym{dnn}{DNN}{Deep Neural Network}
\newacronym{rnn}{RNN}{Recurrent Neural Network}
\newacronym{nn}{NN}{Neural Network}
\newacronym{nlp}{NLP}{Natural Language Processing }
\newacronym{cnn}{CNN}{Convolutional Neural Network}
\newacronym{ann}{ANN}{Artificial Neural Network}

\newacronym{gpu}{GPU}{Graphics Processing Unit}
\newacronym{idsia}{IDSIA}{Istituto Dalle Molle di Studi sull’Intelligenza Artificiale}
\newacronym{ros}{ROS}{Robot Operating System}
\newacronym{ik}{IK}{Inverse Kinematics}
\newacronym{3d}{3D}{three-dimensional}
\newacronym{2d}{2D}{two-dimensional}
\newacronym{1d}{1D}{one-dimensional}

\newacronym{ev}{EV}{Expected value}

\newacronym{led}{LED}{Light Emitting Diode}
\newacronym{rgb}{RGB}{Red Green Blue}

\newacronym{ir}{IR}{Infrared}
\newacronym{gui}{GUI}{Graphical User Interface}
\newacronym{mas}{MAS}{Multi-Agent System}
\newacronym{dai}{DAI}{Distributed Artificial Intelligence}
\newacronym{mal}{MAL}{Multi-Agent Learning}
\newacronym{ma}{MA}{Multi-Agent}
\newacronym{pid}{PID}{Proportional Integral Derivative}
\newacronym{decpomdp}{Dec-POMDP}{Decentralised partially observable Markov 
decision process}


\newacronym{sst}{SST}{Sum of Squares Total}
\newacronym{ssr}{SSR}{sum of squares of the Regression}
\newacronym{sse}{SSE}{Sum of Squared estimate of Errors}
\newacronym{mse}{MSE}{Mean Squared Error}
\newacronym{rmse}{RMSE}{Root Mean Squared Error}
\newacronym{bce}{BCE}{Binary Cross Entropy}
\newacronym{ce}{CE}{Cross Entropy}
\newacronym{auc}{AUC}{Area Under the ROC Curve}
\newacronym{roc}{ROC}{Receiver Operating Characteristic}
\newacronym{r2}{$R^2$}{R Squared}

\newacronym{pc}{PC}{Principal Component}
\newacronym{pca}{PCA}{Principal Component Analysis}
\newacronym{uav}{UAV}{Unmanned Aerial Vehicles}
\newacronym{urdf}{URDF}{Unified Robot Description Format}

\newacronym{tpr}{TPR}{True Positive Rate}
\newacronym{fpr}{FPR}{False Positive Rate}

\newacronym{vr}{VR}{Virtual Reality}
\newacronym{ar}{AR}{Augmented Reality}

\newacronym{poc}{POC}{Proof of Concept}

\newcommand{\footlabel}[2]{%
	\addtocounter{footnote}{1}%
	\footnotetext[\thefootnote]{%
		\addtocounter{footnote}{-1}%
		\refstepcounter{footnote}\label{#1}%
		#2%
	}%
	$^{\ref{#1}}$%
}

\newcommand{\footref}[1]{%
	$^{\ref{#1}}$%
}

\lstdefinelanguage{algebra}
{morekeywords={import,sort,constructors,observers,transformers,axioms,if,
else,end},
sensitive=false,
morecomment=[l]{//s},
}

\title{A webcam-based human-computer interface for defining smooth 3D trajectories by tracking 2D hand landmarks} 
%compulsory
%\specialization{Dependable Distributed Systems}%optional
%\subtitle{Subtitle: Reinventing the World} %optional 
\author{Umberto Cocca} %compulsory
\begin{committee}
\advisor{Dr.}{Alessandro}{Giusti} %compulsory
\coadvisor{Dr.}{Loris}{Roveda}{} %optional
\coadvisor{Dr.}{Gianluigi}{Ciocca}{} %optional
\end{committee}
\Day{23} %compulsory
\Month{February} %compulsory
\Year{2022} %compulsory, put only the year
\place{Lugano} %compulsory

\dedication{-dedica-} %optional
\openepigraph{``Gutta cavat lapidem.'' \\ lat. «la goccia scava la pietra»}
%optional

%\makeindex %optional, also comment out \theindex at the end

\begin{document}

\thispagestyle{empty}
\begin{titlepage}
	
	\noindent
	\begin{minipage}[t]{0.18\textwidth}
		\vspace{-4.5mm}{\includegraphics[scale=0.2]{images/logo_unimib.png}}
	\end{minipage}
	\begin{minipage}[t]{0.82\textwidth}
		{
			\setstretch{1.42}
			{\textsc{Università degli Studi di Milano - Bicocca}} \\
			\textbf{Scuola di Scienze} \\
			\textbf{Dipartimento di Informatica, Sistemistica e Comunicazione} \\
			\textbf{Corso di Laurea Magistrale in Informatica} \\
			\par
		}
	\end{minipage}
	
	\vspace{35mm}
	
	\begin{center}
		{\LARGE{
				\setstretch{1.2}
				\textbf{A webcam-based human-computer interface for defining smooth 3D trajectories by tracking 2D hand landmarks}
				\par
		}}
	\end{center}
	
	\vspace{35mm}
	
	\noindent
	{\large \textbf{Relatore:} Dr. Alessandro Giusti } \\
	
	\noindent
	{\large \textbf{Co-relatore:} Dr. Loris Roveda} \\
	
	\noindent
	{\large \textbf{Co-relatore:} Dr. Gianluigi Ciocca}
	
	\vspace{15mm}
	
	\begin{flushright}
		{\large \textbf{Tesi di Laurea Magistrale di:}} \\
		\large{Umberto Cocca} \\
		\large{Matricola 807191} 
	\end{flushright}
	
	\vspace{10mm}
	\begin{center}
		{\large{\bf Anno Accademico 2020-2021}}
	\end{center}
	
	\restoregeometry
	
\end{titlepage}

\let\cleardoublepage\clearpage
\frontmatter %generates the frontmatter, this is FIXED
%\let\cleardoublepage\clearpage
\begingroup
%\let\cleardoublepage\clearpage
\begin{abstract}
%\addcontentsline{toc}{chapter}{Abstract}  % added 
Robotic systems are increasingly being adopted for filming and photography purposes. In fact, by exploiting robots, it is possible to program complex motions, achieving high-quality videos and photographs. Our contribution is to propose an alternative to the joystick: being able to control a drone using just a hand and giving space to new ways of shooting video content, even to those who are novices. To achieve this, after constructing a Deep Neural Network, which recognizes the gestures of a hand, and later creating a solid pipeline, which detect 3D trajectories obtained from 2D reference points, we captured 3D movements using a state-of-the-art hand tracking system. We estimated the orientation of the hand, and, with this information, we estimated depth as well. 3D trajectories were interpolated and noise-purified with 3 Ridge Regressions. Firstly, as a proof of concept, the captured trajectory was launched in a simulation on a drone, implemented in the ROS framework, and later in a real drone called DJI Ryze Tello. The whole pipeline can be easily translated into another kind of task, e.g., interaction and communication in AR / VR.
\end{abstract}


\begin{acknowledgements}
%\addcontentsline{toc}{chapter}{Acknowledgements}  % added 
% try to relaborate
% https://prothesiswriter.com/blog/how-to-write-an-acknowledgement-for-a-thesis
 
Working on this thesis has been an enriching experience. I was able to work on a great project that allowed me to put into practice most of the skills acquired in these years of university. I understood how complicated it is to take small steps forward in any area of knowledge day after day. \\

\noindent I am glad to have worked on this thesis and I could not ask for better regarding the support of my advisors: Dr. Alessandro Giusti, Dr. Loris Roveda and Dr. Gianluigi Ciocca. They consistently allowed this paper to be my own work but steered me in the right direction whenever they thought I needed it. \\

\noindent Finally, I must express my very profound gratitude to my parents, my girlfriend Maria Grazia and my closest friends Andrea, Salvatore, Guido, Giada, Claudia, Silvia, Leyla and Ivo for providing me with unfailing support and continuous encouragement. This accomplishment would not have been possible without them. Thank you.
\end{acknowledgements}
\endgroup

\tableofcontents 
\newpage

\begingroup
\let\cleardoublepage\clearpage
\listoffigures %optional
\addcontentsline{toc}{chapter}{List of Figures}
\clearpage%\cleardoublepage
\listoftables %optional
\addcontentsline{toc}{chapter}{List of Tables}
%\cleardoublepage
\listof{Equation}{\listequationsname}
\addcontentsline{toc}{chapter}{List of \listequationsname}
\lstlistoflistings
\addcontentsline{toc}{chapter}{List of Listings}
\endgroup
%\let\cleardoublepage\relax

\makeatletter
\renewcommand\mainmatter{\clearpage\@mainmattertrue\pagenumbering{arabic}}
\makeatother

\mainmatter

\begingroup
\let\cleardoublepage\clearpage
\chapter*{Introduction}
\addcontentsline{toc}{chapter}{Introduction}
\markboth{}{}
\label{chap:intro}

% INSERIRE IMMAGINE
% FARE IN NERETTO DOVE C'è DA FARE LISTE

% https://roboticsandautomationnews.com/2021/12/30/robots-increasingly-taking-over-film-camera-work/48012/
Robots are increasingly taking over from what has traditional been the preserve of human creative people. Film creators are progressively turning to automated systems for recording moving images. Producing new, targeted, and engaging content can be a stressful, time-consuming endeavor. This is why more and more creators and advertisers are embracing tools that take the sting out of the process. Cinema robot not only decreases production time, but it allows teams to shoot a variety of angles that would be physically impossible or cost a fortune in labor. \\

\begin{figure}[H]
	\centering
	\includegraphics[width=.7\textwidth]{images/sisujoy}
	\caption[Guide a robot with the motion of the hand using a joystick.]{Guide a robot with the motion of the hand using a joystick.}
	\label{fig:drnsisu}
\end{figure}

% https://www.youtube.com/watch?v=CdOaGU04MyM&t=232s
\noindent Unfortunately, it is known that robot programming may be a very laborious task. They have to synchronize robot motion with objects in the frame. There are also other requirements that need to be taken into account, such as camera settings (focal length, aperture time and desired depth of field). The credit for trying to speed up the process of robot programming goes to different companies. For example, KUKA, a partner of andyRobot, has developed a plug-in for industry leading animation software, Autodesk Maya, that allows KUKA robots to be programmed by anyone who knows how to animate inside Maya \cite[]{Animatin29:online}: a robot program can be created by simply dragging the \gls{3d} model of the robot through space in the virtual world and setting keyframes. Furthermore, SISU Cinema Robotics can be named: the company has developed a way to guide the robot with the motion of the hand using a joystick \cite[]{SISUCine24:online} (see Fig. \ref{fig:drnsisu}). \\

\begin{figure}[H]
	\centering
	\includegraphics[width=.7\textwidth]{images/drn}
	\caption[Drones in filmmaking.]{Drones in filmmaking.}
	\label{fig:drnfilm}
\end{figure}

% https://dakona.co.uk/top-tips-for-drone-cinematography/
\noindent Drone filming and aerial photography (see Fig. \ref{fig:drnfilm}) present a wide range of benefits and opportunities to businesses and brands looking to capture stunning aerial imagery and to produce incredible new perspectives \cite[]{TopTipsf86:online}. There can be quite a steep initial learning curve when it comes to getting to grips with drone technology. Furthermore, to capture high-quality content, it is important to use equipment that is both easy to operate and correctly set up as it is also important to understand how to fly smoothly and consistently capture professional and smooth looking aerial footage. In fact, not every position of the drone are visually interesting angles, as sometimes it is about understanding composition and depth that makes all of the difference. For the success of the video shooting, it is extremely crucial to look for a specific angle that seems unique, cinematic, and suggestive that will hold the attention of the audience. A lot of drones have additional features that help pilots capture great looking footage, such as centre points that help frame shots. Additionally, filters can be added to certain drone cameras, allowing them to control the amount of light entering the lens without compensating with shutter speed. \\

\noindent Our goal is being able to control a drone using just the motion of a hand for shooting, without the use of a joystick. The system can also be used to program a trajectory easily. This can be very interesting because it is an intuitive system: basically, no experience is required in piloting. In addition, the one who shoots could potentially also be the actor. In spite of that, ease of use does not imply that the system does not require attention: it is quite simple to get caught up in the task at hand, and completely lose track of the flight time. Some drones, such as DJI, will begin automatically landing when the battery begins to get low, however there are still a lot of other drones that do not have this intelligence and will literally crash out of the sky when it runs out of battery. Most crashes are actually caused by battery voltage drop \cite[]{WhatHapp70:online}. Not moderating the flight time and not keeping a close eye on the battery levels, can lead to disastrous accidents \cite[]{ListofUA81:online}. \\

\noindent In order to achieve this goal, research has been carried out in the literature to see if a hand tracking system already exists. There is some solution such as \cite[]{oikonomidis2011efficient} where, with the extensive experiments with a prototype GPU-based implementation of the proposed method, demonstrates that accurate and robust \gls{3d} tracking of hand articulations can be achieved in near real-time (15Hz), using a Kinect sensor. A more recent work \cite[]{OculusCo32:online} marking another important \gls{vr} input milestone, in the evolution of \gls{vr} input, is the announcement of hand tracking on Oculus Quest enabling natural interaction in using hands on an all-in-one standalone device. All without the need for a controller, external sensors, gloves, or a PC to power it. And finally, one of the latest research of Google Group research \cite[]{zhang2020mediapipe}: a solution that does not require any additional hardware and performs in real-time on mobile devices (the one actually used in this project and widely spoken in the literature chapter \ref{chap:stateoftheart}). Thanks to this system it was possible to build a \gls{dnn} to recognize specific hand gestures. The orientation of a specific gesture (called "detect gesture") was also estimated in order to obtain also information on depth. Through hand gesture recognition and orientation estimation, a pipeline to detect a \gls{3d} trajectory was defined. However, this trajectory $G()$ was disturbed by estimation errors. Hence, two different data fitting algorithms were applied on the trajectory $G()$: first $3$ splines and then $3$ ridge regression, in both cases for each component $X$,$Y$ and $Z$ of $G()$. This phase allowed to go from trajectory $G()$ to $G_{smooth}()$. \\

\noindent As a \gls{poc}, the captured trajectory was launched in simulation on a drone, implemented in the \gls{ros} framework. After that, the DJI Ryze Tello drone (see Sec. \ref{subsec:tello}) was used to test the application in real life. DJITelloPy is the DJI Tello drone python interface used in the project to make the application communicating with the drone. The trajectory veracity was visible to the naked eye. The correctness of the fixed dimension of the trajectory was extremely important for the project’s purpose since, if the range of action in which the drone operates is uncontrolled, then it could get dangerous for those around it. \\

\noindent The project's objective was reached with success and the entire pipeline for the acquisition of the trajectory can be exploited in \gls{ar} and \gls{vr} scenarios as it would allow people to interact with virtual objects and/or perform actions.
\setcounter{chapter}{0}
\chapter{Literature review}
\label{chap:stateoftheart}
This chapter discusses previous research about the topic, providing a brief 
introduction to the approach we adopted.
%, evaluating, in particular, their shortcomings (difetti). 

\glsreset{ai}
\glsreset{dl}
\glsreset{rl} 
\glsreset{dnn}

\bigskip
Siccome non è mai esistito un sistema che riconosce la mano davvero efficiente e in tmepo reale allora non è mai esistito la possibilità di poter controllare le traiettorie con la mano. In seguito alla soluzione recente fatta dai ricercatori di google che ha risolto i diversi problemi di performance e precisione nel riconoscere la mano, ora è possibile ragionare su come eseguire delle traiettorie con la mano. QUesto mio studio ha l'obiettivo quindi di aggiungere un'ulteriore mattoncino al lavoro svolto dai ragazzi di google per controllare il drone usando soltanto la mano, risolvendo quindi diversi problemi.
\chapter{Background}
\label{chap:background}
\glsreset{nn}
%FIXME sections
This chapter provides some background concepts to understand the material 
presented in this thesis. 

%Nel caso la tesi riguardi un progetto di sviluppo software,
%è preferibile, per maggiore chiarezza, descrivere i requisiti, l’interfaccia utente e
%l’architettura in capitoli separati (il codice potrà essere allegato in appendice). Il capitolo
%finale dovrà contenere una sintesi del lavoro, e una descrizione degli eventuali problemi
%aperti e dei possibili sviluppi futuri.

\section{Regression}
% Bishop -pattern recognition and machine learning (pg. 137)
% andrew ng - Da pag 3 fino a pag 11 (cioè Polynomial Regression)
The goal of regression is to predict the value of one or more continuous target variables $t$ given the value of a $D$-dimensional vector $x$ of input variables. \cite[]{bishop:2006:PRML}

\noindent Given a training data set comprising $N$ observations ${x_n}$, where $n = 1,...,N$,
together with corresponding target values ${t_n}$, the goal is to predict the value of $t$
for a new value of $x$.

\noindent From a probabilistic perspective, the aim is to model the predictive distribution $p(t|x)$ because this expresses the uncertainty about the value of $t$ for each value of $x$.

\subsection{Linear Models for Regression}
\label{subsec:reglinuniv}

The simplest linear model for regression is one that involved a linear combination of the input variables:
\begin{Equation}[H]
	\centering
	\begin{equation} \label{eq:lincomb}
		y(x,w)=w_0 + w_1x_1+...w_D x_D
	\end{equation}
\end{Equation}

\noindent where $x=(x_1,...,x_D)^T$. This is often simply known as linear regression. The key property of this model is that it is a linear function of the parameters $w_i$, but also of $x_i$ and establishes significant limitations on the model. It is possible extend the class of models by considering linear combinations of fixed nonlinear functions of the input variables:
\begin{Equation}[H]
	\centering
	\begin{equation} \label{eq:lincombbasis}
		y(x,w)=w_0 + \sum_{j=1}^{M-1}w_j \phi_j(x)
	\end{equation}
\end{Equation}

\noindent where $\phi_j(x)$ are known as basis functions. The total number of parameters in this model will be $M$.

\noindent The parameter $w_0$ is called bias parameter. It is often convenient to define an additional dummy "basis function" $\phi_0(x)=1$, so that:

\begin{Equation}[H]
	\centering
	\begin{equation} \label{eq:vectlincomb}
		y(x,w)=\sum_{j=0}^{M-1}w_j \phi_j(x) = w^T\phi(x)
	\end{equation}
	\caption[Linear combinations of fixed nonlinear basis functions.]{By using non linear basis functions, the function $y(x,w)$ can be a non linear function of the input vector $x$.}
\end{Equation}

\noindent where $w=(w_0,...,w_{M-1})$ and $\phi=(\phi_0,...,\phi_{M-1})^T$. \\

\noindent A particle example of this model where there is a single input variable $x$ is the polynomial regression. The basis functions take the form of powers of $x$ so that $\phi_j(x)=x^j$.

\noindent There are other possible choices for the basis functions as:

\begin{Equation}[H]
	\centering
	\begin{equation} \label{eq:gausbasfun}
		\resizebox{.25\hsize}{!}{$ \phi_j(x)=e^{\frac{-(x-u_j)^2}{2s^2}} $}
	\end{equation}
	\caption[Gaussian basis function.]{These are usually referred to as "Gaussian" basis functions.}
\end{Equation}

\noindent where $u_j$ regulates the locations of the basis functions in input space, while the parameter $s$ is their spatial scale.
\noindent The identity basis functions in which the vector $\phi(x)=x$ can be simply used.

\subsection{Normal equations}
\label{subsec:reglinmulnormeq}
% Bishop -pattern recognition and machine learning (pg. 25 intro chap 1.1)
%https://en.wikipedia.org/wiki/Mean_squared_error
The values of the coefficients will be determined by fitting the polynomial to the training data. This can be done by minimizing an error function that measures the misfit between the function $y(x, w)$, for any given value of $w$, and the training set data points. One simple choice of error function, which is widely used, is given by the \gls{sse} between the predictions $y(x_n, w)$ for each data point $x_n$ and the corresponding target values $t_n$ (called also sum of squares regression \cite[]{sum-squares}), so that the following is minimised:

\begin{Equation}[H]
	\centering
	\begin{equation} \label{eq:sumsquarereg}
		E(w)=\frac{1}{2} \sum_{n=1}^{N} [y(x_n,w)-t_n]^2
	\end{equation}
	\caption[Sum of squares regression.]{Is a statistical technique used in regression analysis to determine the dispersion of data points and the function that best fits (varies least) from the data.}
\end{Equation}

\noindent where the factor of 1/2 is included for mathematical convenience. It is a nonnegative quantity that would be zero if, and only if, the function $y(x, w)$ were to pass exactly through each training data point.

\noindent The curve fitting problem can be solved by choosing the value of $w$ for which $E(w)$ is as small as possible. Because the error function is a quadratic function of the coefficients $w$, its derivatives with respect to the coefficients will be linear in the elements of $w$, and so the minimization of the error function has a unique solution $w^*$. The resulting polynomial is given by the function $y(x, w^*)$.

% Leggere gli articoli che chiariscono differenza tra MSE e RSS...
%https://stats.stackexchange.com/questions/73540/what-is-the-relationship-between-the-mean-squared-error-and-the-residual-sum-of
%https://365datascience.com/tutorials/statistics-tutorials/sum-squares/
%https://en.wikipedia.org/wiki/Errors_and_residuals#Regressions

% http://michael.orlitzky.com/articles/the_derivative_of_a_quadratic_form.xhtml
% derivative_of_a_quadratic_form

% Derivative of transpose of a matrix
% https://en.wikipedia.org/wiki/Matrix_calculus#Identities_in_differential_form

% Derivate of matrix form
% https://math.stackexchange.com/questions/756679/least-squares-residual-sum-of-squares-in-closed-form/757130#757130
\noindent The formula (\ref{eq:sumsquarereg}) can be written in matrix notation as:

\begin{Equation}[H]
	\centering
	\begin{equation} \label{eq:sumsquareregvec}
	\frac{1}{2} (\phi w - t)^T (\phi w - t)
	\end{equation}
	\caption[Sum of squares regression in matrix notation.]{This is the sum of squares regression in matrix notation.}
\end{Equation}

\noindent where $\phi$ is an $N x M$ matrix, so that:
 
% https://jasonwarta.github.io/latex-matrix/ 
\begin{Equation}[!htb]
	\centering
	\begin{equation} \label{eq:designmatrix}
	\phi =
		\begin{pmatrix}
			\phi_0(x_1) & \phi_1(x_1) & \dots & \phi_{M-1}(x_1) \\
			\phi_0(x_2) & \phi_1(x_1) & \dots & \phi_{M-1}(x_1) \\
			\vdots & \vdots & \ddots & \vdots \\
			\phi_0(x_N) & \phi_1(x_N) & \dots & \phi_{M-1}(x_N) \\
		\end{pmatrix}
	\end{equation}
	\caption[Design matrix.]{This is called the design matrix whose elements are given by $\phi_{nj} = \phi_j(x_n)$.}
\end{Equation}

\begin{Equation}[!htb]
	\centering
	\begin{equation} \label{eq:weightsandtarget}
		w = 
		\begin{pmatrix}
			w_0 \\
			\vdots \\
			w_{M-1}
		\end{pmatrix}
		\quad t =
		\begin{pmatrix}
			t_1 \\
			\vdots \\
			t_N \\
		\end{pmatrix}
	\end{equation}
	\caption[Weights of $M-1$ features and target of $N$ examples.]{Weights of $M-1$ features and target of $N$ examples.}
\end{Equation}

\noindent Therefore, the optimization problem has to be solved finding the minimum of the cost function $E(w)$:
\begin{Equation}[H]
	\centering
	\begin{equation} \label{eq:mincostregr}
		w^*= \operatorname*{arg\,min}_w  \frac{1}{2}(\phi w - t)^T (\phi w - t)
	\end{equation}
	\caption[Optimization problem for ridge regression.]{The goal is to find the weight $w$ that minimize the cost function $E(w)$.}
\end{Equation}

\noindent The gradient is computed and solving for $w$ the result obtained is:
\begin{Equation}[H]
	\centering
	\begin{equation} \label{eq:normeq}
		\begin{aligned}
			\nabla E(w) = \phi^T(\phi w - t) &= 0 \\
			\phi^T \phi w - \phi^T &= 0 \\
			\phi^T \phi w &= \phi^T t \\
			w^* &= (\phi^T \phi)^{-1} \phi^T t
		\end{aligned}
	\end{equation}
	\caption[Normal equations.]{They are known as the normal equations for the least squares problem}
\end{Equation}

\noindent This is the real minimum because if the second derivative $\nabla^2 E(w) = \phi^T \phi$ is taken, this is a symmetric matrix, so it's also positive definitive matrix, which means that this objective function that is being minimized is convex. So if a stationary point is found, such that derivative is zero, a global minimum is also found. Furthermore, to find a solution the matrix $\phi^T \phi$ needs to be inverted, so some condition that assures this is invertible is needed, and this is the case when the columns of the matrix are linearly independent. \\

\noindent Once the solution $w^*$ is found and a new data point that has never been seen during training is received, the new target $t^*$ is predicted as:

\begin{Equation}[H]
	\centering
	\begin{equation} \label{eq:soloptridgereg}
		\begin{aligned}
			t^* = w^{*^T} \phi(x)
		\end{aligned}
	\end{equation}
	\caption[Prediction in ridge regression.]{The goal is to find the weight $w$ that minimize the cost function $E(w)$.}
\end{Equation}

\section{The Problem Of Overfitting}
\label{subsec:poverfitting}
If an expensive set of features is used (for example a ten polynomial grade $x^{10}$), then the model interpolates very close the training data, because there is a model with $10$ parameters where the data points can be perfectly represented. The risk is that the model overfit data points (see Fig. \ref{fig:overunderfit} (center)).

\begin{figure}[H]
	\begin{minipage}{\textwidth}
		\centering
		\includegraphics[width=.95\textwidth]{images/overunderfit}
		\caption[Overfitting and Underfitting.]{Image by "Applied Supervised Learning with R" book \footnote{\url{https://subscription.packtpub.com/book/data/9781838556334/7/ch07lvl1sec82/underfitting-and-overfitting}}. It is possible to see high bias resulting in an oversimplified model (that is underfitting); high variance resulting in overcomplicated models (that is overfitting); and lastly, striking the right balance between bias and variance.}
		\label{fig:overunderfit}
	\end{minipage}
\end{figure}

\noindent On the other hand, if the model is not too expressive and not too complex our data will be linearly representable in the feature space and this means that the performance will be very poor (see Fig. \ref{fig:overunderfit} (left)). A trade-off between fits data and being able to generalize (see Fig. \ref{fig:overunderfit} (center)) is desired.

\subsection{Regularization}
Some features with high values parameters have been penalized, if the model is overfitting, it is very likely that its parameters will have a big magnitude. This means that also the features that supply to these parameters will be higher and very low and this can cause a lot of problems. \\

\noindent In order to control over-fitting the idea of adding a regularization term to an error function is introduced, so that the total error function to minimized takes the form:

\begin{Equation}[H]
	\centering
	\begin{equation}\label{eq:fullreg}
		\begin{aligned}
			E(w) = E_D(w) + \lambda E_W(w)
		\end{aligned}
	\end{equation}
\end{Equation}

\noindent Where $\lambda$ is the regularization coefficient and it is the trade-off between how well fit training set is and how to establish the parameters $w$ with low values, therefore having simple hypotesys avoiding over-fitting. $E_D$ is the error based on dataset, while $E_W$ is based on weights.

\subsection{Cost Function}
\label{subsec:regcostfun}

One of the simplest forms of regularizer is given by the sum-of-squares of the weight vector elements:
\begin{Equation}[H]
	\centering
	\begin{equation}\label{eq:regularizer}
		\begin{aligned}
			E_W(w) = \frac{1}{2} w^T w = \frac{||w||^2_2}{2}
		\end{aligned}
	\end{equation}
	\caption[Cost function | Regularisation term.]{Cost function | Regularisation term.}
\end{Equation}

\noindent where $||w||_2$ is the euclidean norm $\sqrt{ \sum_{i=1}^{n} x^2_i}$. \\

%https://en.wikipedia.org/wiki/Ridge_regression
\noindent This is also called ridge regression, a method of estimating the coefficients of multiple-regression models in scenarios where independent variables are highly correlated.

\noindent If the sum-of-squares error function given by the following formula is considered:
\begin{Equation}[H]
	\centering
		\begin{equation} \label{eq:sumsqinverted}
			\begin{aligned}
				E_D(w) = \frac{1}{2} [t_n - w^T \phi(x_n)]^2
			\end{aligned}
		\end{equation}
	
\end{Equation}

\noindent then the total error function becomes:
\begin{Equation}[H]
	\centering
		\begin{equation} \label{eq:lossridgereg}
			\begin{aligned}
				E(w) = \frac{1}{2} \sum_{n=1}^{N}[t_n - w^T \phi(x_n)]^2 + \frac{\lambda}{2} w^T w
			\end{aligned}
		\end{equation}
		\caption[Loss Function for Ridge Regression]{This particular choice of regularizer is known in the machine learning literature as weight decay because in sequential learning algorithms, it encourages weight values to decay towards zero.}
\end{Equation}

\noindent Setting the gradient of $E(w)$ \gls{wrt} $w$ to zero, and solving for $w$, the result obtained is:

\begin{Equation}[H]
	\centering
		\begin{equation} \label{eq:least-square-stepf}
			\begin{aligned}
				w^* = (\lambda I + \phi^T \phi)^{-1} \phi^T t
			\end{aligned}
		\end{equation}
		\caption[Ridge regression solution]{This is an extension of the least-squares solution \ref{eq:normeq}.}
\end{Equation}

\noindent From this side it denotes a better version than before because is also possible prove that ($\lambda I + \phi^T \phi$) is always invertible if $\lambda > 0$, therefore $w^*$ always exists.

\subsection{(Batch) Gradient Descent}
\label{subsec:batchgradientdescen}
\noindent Gradient descent is a first-order iterative optimization algorithm for finding the minimum $w$ of a function $E(w)$. To achieve this goal, it performs two steps iteratively, until convergence:

\begin{itemize}
	\item Compute the slope (gradient) that is the first-order derivative of the function at the current point;
	\item Move-in the opposite direction of the slope increase from the current point by the computed amount.
\end{itemize}

%it should edit this pic to avoid copyright...
\begin{figure}[H]
	\centering
	\includegraphics[width=.6\textwidth]{images/grad_desc}
	\caption[Gradient descent.]{Gradient descent is based on the observation that if the multi-variable function $E(w)$ is defined and differentiable in a neighborhood of a point $w_0$ , then $w_0$ decreases fastest if one goes from $w_0$  in the direction of the negative gradient of $E(w)$ at $w_0$, $-\nabla E(w_0)$.}
	\label{fig:grad_desc}
\end{figure}

% https://stats.stackexchange.com/questions/539137/why-use-mse-instead-of-sse-as-cost-function-in-linear-regression
\noindent Let's now use \gls{mse}, instead of using (\ref{eq:sumsquarereg}) due to the fact that it is possible to reach obvious benefits: first of all to keep the value in a expressible range usable by computers, then to make the results comparable across samples regardless of the size of the sample. In fact, the \gls{sse} depends on how many terms are added up (note the case of millions/billions of data points). In addition, using \gls{sse} or \gls{mse} it still leads to find an equivalent solution.

\begin{Equation}[H]
	\centering
	\begin{equation} \label{eq:mser}
		\begin{aligned}
			E_D(w) = \frac{1}{2N} \sum_{n=1}^{N} [y(x_n,w)-t_n]^2
		\end{aligned}
	\end{equation}
	\caption[Mean Squared Error.]{Mean Squared Error. $1/2$ is added, as in \gls{sse}, so the derivative doesn't need a constant out front. The problem is not an issue, because the minima of $E_D(w)$ and $E_D(w) / 2$ are achieved at the same value(s) of $w$.}
\end{Equation}

\noindent Concerning the regularisation term $E_W(w)$ (\ref{eq:regularizer}), it could have been written:
\begin{Equation}[H]
	\centering
	\begin{equation} \label{eq:regnew}
		\begin{aligned}
			E_W(w) = \frac{ \sum_{m=1}^{M} {w^2_m}}{2}
		\end{aligned}
	\end{equation}
	\caption[Regularisation term.]{Regularisation term.}
\end{Equation}

\noindent Note that, conventionally, $m$ starts from $1$, and not from $0$, even if it exists (\ref{eq:weightsandtarget}). Nevertheless, it's important to know that nothing change consistently even if the 0th weight is considered. 

\noindent Combining together following (\ref{eq:fullreg}) 

% WE NEED CHECK M, CHANGE ALPHA WHERE I HAVE SPOKEN ABOUT GRANDIENT
\begin{Equation}[H]
	\centering
	\begin{equation} \label{eq:fullregnew}
	\begin{aligned}
	E(w) = \frac{1}{2N} \left\{ \sum_{n=1}^{N} [ y(x_n,w)-t_n ]^2 + \lambda \sum_{m=1}^{M} {w^2_m} \right\}
	\end{aligned}
	\end{equation}
	\caption{Gradient Descent.}
\end{Equation}

\noindent Have some function $E(w)$ \\
\noindent Want $\operatorname*{arg\,min}_w  E(w))$ \\

\noindent Keep changing $w$ to reduce $E(w)$ until a minimum is hopefully reached:

\begin{Equation}[H]
	\centering
	\begin{equation} \label{eq:graddesc}
		\begin{aligned}
			Repeat = \{ \\
				w_0 &:= w_0 - \alpha \frac{1}{N} \sum_{n=1}^{N} [ y(x_n,w)-t_n ]^2 \, x_0 \\
				w_j &:= w_j - \alpha \frac{1}{N} \sum_{n=1}^{N} [ y(x_n,w)-t_n ]^2 \, x_j + \frac{\lambda}{M} w_j\\
			\}
		\end{aligned}
	\end{equation}
	\caption[Gradient Descent in compact form.]{Cost function for Gradient Descent in compact form.}
\end{Equation}

\noindent where in compact form is:
\begin{Equation}[H]
	\centering
	\begin{equation} \label{eq:graddesccompact}
	\begin{aligned}
		Repeat = \{ \\
			w_j &:= w_j(1-\alpha \frac{\lambda}{N}) - \alpha \frac{1}{N} \sum_{n=1}^{N} [ y(x_n,w)-t_n ]^2 \, x_j \\
		\}
	\end{aligned}
	\end{equation}
	\caption[Cost function for Gradient Descent.]{Cost function for Gradient Descent.}
\end{Equation}



\section{Artificial Neural Networks}
\label{sec:nn}
The term \gls{nn} has its origins in attempts to find mathematical representations of information processing in biological systems. In fact, \glspl{ann}, subset of \gls{ml} field, are models inspired from the biological performance of human brain \cite[]{inbook}. 

\begin{figure}[H]
	\centering
	\includegraphics[width=.7\textwidth]{images/Neuron3}
	\caption[Image of a human neuron.]{Neuron and myelinated axon, with signal flow from inputs at dendrites to outputs at axon terminals. The signal represents a short electrical pulse called 'spike'.}
	\label{fig:bioneuron}
\end{figure}

\noindent Neurons have cell body (fig. \ref{fig:bioneuron}) and a number of input wires called dendrites. Neurons also have an output wire called axon, used to send signals to other neurons. At a simplistic level, the neuron is a computational unit that gets a number of inputs through its input wires, does some computation, and finally it sends outputs to other neurons connected to it in the brain. \\

\subsection{Feedforward Fully-Connected Neural Networks}
\label{nn:feedforward}

\noindent In (fig. \ref{fig:nn}) it is possible to see a \gls{nn}, seen as mathematical model. It is just a group of these different neurons strung together. Trying to underline an analogy with the biological systems: circles identify the cell body where they are fed with some inputs that pass through the input wires, similar to the dendrites. The neuron does some computation and outputs some value on an output wire, where in the biological neuron it identifies the axon.

\begin{figure}[H]
	\centering
	\includegraphics[width=.7\textwidth]{images/NN}
	\caption[Feed forward neural network.]{The image graphically shows a \gls{nn} with three layers (or two hidden layers). The input layer has $M^{(0)}$ neurons and the output layer has $K$ outputs. Neurons are organized in  layers to allow parallel computation to avoid cyclic dependencies. The process of computing \gls{nn} outputs from inputs is called forward propagation. This kind of network is also called Multi-layer perceptron.}
	\label{fig:nn}
\end{figure}

\noindent The $z^{(l)}_m$ are neurons, each takes its input values and computes a single output value from them. Neurons are organized in layers $1,...,L$ and usually the starting input is considered the 0th layer. Inputs $x_1,...,x_D$ are occasionally called input layer/neurons (even though they do not compute anything). The output of the entire network is then $y=z^{(L)}$, called output layer. Instead, the internal layers are called hidden layers. Each layer $l\in \{1,...,L\}$ has $M^{(l)}$ neurons.

\noindent $M^{(1)}$ neurons perform a perceptron-like computation:

\begin{Equation}[H]
	\centering
	\begin{equation} \label{eq:neurbas}
		u^{(1)}_m = (w^{(1)}_m)^T x + b^{(1)}_m,  
		\quad \quad
		z^{(1)}_m = f(u{(1)}_m),
		\quad \quad
		m=1,...,M^{(1)}
	\end{equation}	
\end{Equation}

\noindent with a differentiable activation function $f$ for gradient descent. This step is iterated multiple times taking the outputs of the previous step:

\begin{Equation}[H]
	\centering
	\begin{equation} \label{eq:neurbas2}
		z^{(l-1)} = (z^{(l-1)}_m)_{m=1,...,M^{(l-1)}}
	\end{equation}	
\end{Equation}

\noindent as input of:

\begin{Equation}[H]
	\centering
	\begin{equation} \label{eq:neurbas3}
		u^{(l)}_m = (w^{(l)}_m)^T z^{(l-1)} + b^{(l)}_m,  
		\quad \quad
		z^{(l)}_m = f(u^{(l)}_m)
	\end{equation}
\end{Equation}

\noindent where $m=1,...,M^{(1)}$ and $l=2,...,L$. Weights $w$ are usually independent for each step. Additionally define $z^{(0)}$ to be the input, i.e.

\begin{Equation}[H]
	\centering
	\begin{equation} \label{eq:neurbas4}
		z^{(0)} = x
	\end{equation}
\end{Equation}

\noindent For each layer $l \in 1,...,L$ the computation is

\begin{Equation}[H]
	\centering
	\begin{equation} \label{eq:neurbas5}
		z^{(l)}_m = f( (w^{(l)}_m)^T z^{(l-1)} + b^{(l)}_m)
	\end{equation}
\end{Equation}

\noindent which can be written as a matrix multiplication:

\begin{Equation}[H]
	\centering
	\begin{equation} \label{eq:forwpropag}
		z^{(l)} = f( W^{(l)} z^{(l-1)} + b^{(l)})
	\end{equation}
	\caption[Forward propagation.]{Function that identifies input transformation at each step $l$ of the net.}	
\end{Equation}

\noindent The weights $w$ are directed connections between the neurons, e.g. the neurons of layer 2 are connected to the ones of layer 1 by the weights $w^{(2)}_{mn}, m=1,...M^{(1)}, n=1,...,M^{(2)}$. The bias $b$ varies according to the propensity of the neuron to activate, influencing its output. \\

\noindent $f()$ is an activation function and needs to be differentiable, so that gradient descent training is applicable. For the hidden layers of the network, the activation function must be nonlinear, because multiple linear computations can be collapsed to a single one, therefore in order to gain power from iterative computation, nonlinear steps are needed.

\noindent Many possible activation functions for the hidden layers of a \gls{nn} exist:
\begin{itemize}
	\item Sigmoid, Hyperbolic Tangent: Monotonic, squeeze output to a fixed range
	\item ReLU: "almost linear" (a clipped identity function)
\end{itemize}

\begin{figure}[H]
	\centering
	\includegraphics[width=.9\textwidth]{images/old_act_fun}
	\caption[Image of a human neuron.]{Neuron and myelinated axon, with signal flow from inputs at dendrites to outputs at axon terminals. The signal represents a short electrical pulse called 'spike'.}
	\label{fig:old_act_fun}
\end{figure}

\noindent One of the key characteristics of modern deep learning system is to use non-saturated activation function (e.g. ReLU) to replace its saturated counterpart (e.g. sigmoid, tanh). The advantage of using non-saturated activation function lies in two aspects: the first is to solve the so called "exploding/vanishing gradient" \cite[]{bengio1994learning}, in particular on the difficulty of training \gls{rnn} \cite[]{pascanu2013difficulty}, while the second is to accelerate the convergence speed. More sophisticated activation function as the "leaky ReLU" try to solve the dying ReLU problem \cite[]{LeakyReL95:online}. In contrast to ReLU, in which the negative part is totally dropped, leaky ReLU assigns a non-zero slope to it. Leaky ReLU and its variants are consistently better than ReLU in \gls{cnn} \cite[]{xu2015empirical}.

\noindent Leaky Rectified linear activation is introduced in acoustic model \cite[]{maas2013rectifier}. Mathematically, it is defined as follows:

\begin{Equation}[H]
	\centering
	\begin{equation} \label{eq:leakyrelu}
 		f(x) = 
			\begin{cases}
			\frac{x}{a} & \text{if $x < 0 $} \\
			x & \text{otherwise}
		\end{cases}
	\end{equation}
	\caption[Leaky Rectified linear activation.]{Function that identifies input transformation at each step $l$ of the net.}
\end{Equation}

\noindent where $a$ is a fixed parameter in range $(1; +\inf)$. In original paper, the authors suggest to set $a$ to a large number like $100$.

\begin{figure}[H]
	\centering
	\includegraphics[width=.65\textwidth]{images/leaky_relu}
	\caption[Leaky Rectified linear activation.]{Neuron and myelinated axon, with signal flow from inputs at dendrites to outputs at axon terminals. The signal represents a short electrical pulse called 'spike'.}
	\label{fig:leakyrelu}
\end{figure}

\subsection{\gls{nn} Setup for Classification}
\label{nn:nnclassification}
For a classification task with $K$ classes, a $K$-dimensional output layer is used. A sample $x \in R^D$ is classified as belonging to class $k$ if the output neuron $y_k$ has the maximal value:

\begin{Equation}[H]
	\centering
	\begin{equation} \label{eq:nnclass}
		c^*= \operatorname*{arg\,max}_k  y_k
	\end{equation}
\end{Equation}

\noindent The problem is that the $arg max$ function is not differentiable. Therefore, this is solved by letting the \gls{nn} output a probability distribution over classes:

\begin{Equation}[H]
	\centering
	\begin{equation} \label{eq:nnclass2}
		y = (y_k)_{k=1,...,K}
		\quad
		\quad
		\quad
		y_k \geq 0,
		\quad
		\sum_{k} y_k = 1
	\end{equation}
\end{Equation}

\noindent The advantage is that a differentiable measure of the quality of the output on theoretical grounds is derivable, using probability theory. In order to make the network output a probability distribution, exponentials are taken and normalized. This is the softmax nonlinearity:

\begin{Equation}[H]
	\centering
	\begin{equation} \label{eq:softmax}
		S(y) = (\frac{e^{y_1}}{\sum_k e^{y_k}},...,\frac{e^{y_K}}{\sum_k e^{y_k}})
	\end{equation}
	\caption[Softmax nonlinearity.]{Softmax nonlinearity.}
\end{Equation}

\noindent In contrast to other activation functions, it is applied to the full last layer of the \gls{nn}, not to each indipendent component. The hidden layers can have any nonlinear activation function.

\noindent The learning process is structured as a non-convex optimisation problem in which the aim is to minimise a cost function, which measures the distance between a particular solution and an optimal one.

\noindent If a \gls{nn} with softmax output is assumed, can be computed the loss by measuring the cross-entropy between the output distribution and the target distribution.

\noindent Encoding the target in one-hot style, e.g. if a sample belongs to class k, the target is:

\begin{Equation}[H]
	\centering
	\begin{equation} \label{eq:nnclass3}
		t=(0,...,0,1,0,...,0)
	\end{equation}
\end{Equation}

\noindent This is treated as a probability distribution: in an ideal world, a perfect hypothesis $y$ would exactly match this $t$, assigning probability $1$ to the correct class, and probability $0$ otherwise. The cross-entropy loss is defined as:

\begin{Equation}[H]
	\centering
	\begin{equation} \label{eq:cross-entropy}
		E_{CE} = -\sum_{k}(t_k \, \log y_k)
	\end{equation}
	\caption[Cross-entropy loss.]{Cross-entropy loss.}
\end{Equation}

\noindent The intuition about the cross-entropy corresponds to the number of additional bits needed to encode the correct output, given that the (possibly wrong) prediction of the network is accessible. One property of the cross-entropy loss is that is always non-negative. For efficiency and numerical stability, one should merge softmax loss and cross-entropy criterion into one function:

\begin{Equation}[H]
	\centering
	\begin{equation}
		E_{CE+SM} = -\sum_{k}(t_k \, \log S_k(y))
	\end{equation}
	\caption[Softmax cross entropy.]{To train the network with backpropagation, the calculation of the derivative of the loss is needed. In the general case, that derivative can get complicated, but using the softmax and the cross entropy loss, that complexity fades away.}
	\label{eq:softmaxcrossentr}
\end{Equation}

\subsection{Optimisation algorithms}
\label{nn:optmalgo}
The choice of optimization algorithms strongly influences the effectiveness of the learning process, as they update and calculate the appropriate and optimal values of that model. Specifically, if the gradient descent is considered, which is the most popular optimization strategy used in machine learning, the extent of the update is determined by the learning rate $\lambda$, which guarantees convergence to the global minimum for convex error surfaces and to a local minimum for non-convex surfaces. By the way, there are better optimization as the non linear conjugate gradient \cite[]{conjugategradient}, BFGS, the improved version to decrease memory usage L-BFGS \cite[]{saputro2017limited}, etc... the main advantages are that there is no need to manually pick $\lambda$ and often are faster than gradient descent, although more complex algorithms. \\

\noindent The optimiser chosen for this thesis project is Adam, an algorithm for first-order gradient-based optimisation of stochastic objective functions, based on adaptive estimates of lower-order moments \cite[]{kingma2017adam}. It is an extension to stochastic gradient descent that has recently seen broader adoption for \gls{dl} applications in computer vision and \gls{nlp}.

\begin{figure}[H]
	\begin{minipage}{\textwidth}
		\centering
		\includegraphics[width=.6\textwidth]{images/adam}
		\caption[Comparison of Adam to other optimization algorithms.]{Comparison of Adam to other optimization algorithms training a Multilayer Perceptron\footnote{Taken from Adam: A Method for Stochastic Optimization, 2015}.}
		\label{fig:adam}
	\end{minipage}
\end{figure}
\chapter{Tools}
\label{chap:impl}
\glsreset{ir}

This chapter introduces the tools used in this work, starting from the description of the target platform in, then describing the simulator in and finally mentioning the frameworks used for the implementation in.

\section{Tello}
\label{subsec:tello}
Tello is a small quadcopter that features a Vision Positioning System and an onboard camera. Using its advanced flight controller, it can hover in place and is suitable for flying indoors. Tello caputers 5MP photos and stream until 720p live video. Its maximum flight time is approximately 12 minutes (tested in windless conditions at a consistent $15km/h$) and its maximum flight distance id $100m$ \cite[]{djitelloguide}.

\begin{figure}[H]
	\centering
	\includegraphics[width=.7\textwidth]{images/tello}
	\caption[Tello - Aircraft diagram.]{1.Propellers; 2.Motors; 3.Aircraft Status Indicator; 4.Camera; 5.Power Button; 6.Antennas; 7.Vision Positioning System; 8.Flight Battery; 9.Micro USB Port; 10.Propeller Guards.}
	\label{fig:telloairdiagr}
\end{figure}

\noindent The Tello can be controlled manually using the virtual joysticks in the Tello app or using a compatible remote controller. It also has various Intelligent Flight Modes that be used to make Tello perform maneuvers automatically. Propeller Guards can be used to reduce the risk of harm or damage people or objects resulting from accidental collisions with Tello aircraft.

\subsection{Tello Command Types and Results}
\label{subsec:tellosdk}
The Tello SDK connects to the aircraft through a Wi-Fi UDP port, allowing users to control the aircraft with text commands. There are Control and Set commands where return "ok" if the command was successful or "error" or an informational result code if the ocmmand failed. There are also Read commands that return the current value of the sub-parameters.

% Please add the following required packages to your document preamble:
% \usepackage{booktabs}
% \usepackage{multirow}
\begin{table}[H]
	\begin{tabular}{@{}|llc|@{}}
		\toprule
		\multicolumn{3}{|c|}{\textbf{Main Tello Commands}}                                                                                                                                                                                                                                                                                                                      \\ \midrule
		\multicolumn{1}{|l|}{\textbf{Command}}  & \multicolumn{1}{l|}{\textbf{Description}}                                                                                                                                                                                                                                   & \multicolumn{1}{l|}{\textbf{Possible Response}} \\ \midrule
		\multicolumn{1}{|l|}{connect}           & \multicolumn{1}{l|}{Enter SDK mode.}                                                                                                                                                                                                                                        & \multirow{6}{*}{ok / error}                     \\ \cmidrule(r){1-2}
		\multicolumn{1}{|l|}{streamon}          & \multicolumn{1}{l|}{Turn on video streaming.}                                                                                                                                                                                                                               &                                                 \\ \cmidrule(r){1-2}
		\multicolumn{1}{|l|}{streamoff}         & \multicolumn{1}{l|}{Turn off video streaming.}                                                                                                                                                                                                                              &                                                 \\ \cmidrule(r){1-2}
		\multicolumn{1}{|l|}{takeoff}           & \multicolumn{1}{l|}{Auto takeoff.}                                                                                                                                                                                                                                          &                                                 \\ \cmidrule(r){1-2}
		\multicolumn{1}{|l|}{land}              & \multicolumn{1}{l|}{Auto landing.}                                                                                                                                                                                                                                          &                                                 \\ \cmidrule(r){1-2}
		\multicolumn{1}{|l|}{send\_rc\_control} & \multicolumn{1}{l|}{\begin{tabular}[c]{@{}l@{}}Set remote control via four channels.\\ Arguments:\\ - left / right velocity: from -100 to +100\\ - forward / backward velocity: from -100 to +100\\ - up / down: from -100 to +100\\ - yaw: from -100 to +100\end{tabular}} &                                                 \\ \midrule
		\multicolumn{1}{|l|}{get\_battery}      & \multicolumn{1}{l|}{Get current battery percentage}                                                                                                                                                                                                                         & \multicolumn{1}{l|}{from 0 to +100}             \\ \bottomrule
	\end{tabular}
	\captionof{table}[Tello Python Commands.]{List of the main Tello functions of the python wrapper to interact with the Ryze Tello drone using the official Tello api.}
	\label{tab:modeln5dist}
\end{table}

\section{Gazebo}
\label{subsec:gazebo}

% parlare che si è lavorato su windows, ma gazebo era in una macchina virtuale con immagine ubuntu.
Gazebo is a 3D simulator, offers the ability to accurately and efficiently simulate robots in complex indoor and outdoor environments. Thanks to Gazebo it was possible to launch the 3D trajectory acquired by hand through the webcam on a simulated drone. 

\section{Frameworks}
\label{subsec:frameworks}

\paragraph*{\texttt{DJITelloPy}} DJI Tello drone python interface using the official Tello SDK and Tello EDU SDK. This library has an implementation of all tello commands, easily retrieve a video stream, receive and parse state packets and other features.\footnote{\url{https://github.com/damiafuentes/DJITelloPy}}.

\paragraph*{\texttt{TensorFlow}} is an end-to-end open source platform for \gls{ml}. It has a comprehensive, flexible ecosystem of tools, libraries and community resources that lets researchers push the state-of-the-art in \gls{ml}\footnote{\url{https://www.tensorflow.org/}}.

\paragraph*{\texttt{NumPy}} is a highly optimized library for scientific computing that provides support for a range of utilities for numerical operations with a MATLAB-style syntax. manipulation\footnote{\url{https://numpy.org}}.

\paragraph*{\texttt{OpenCV-Python}} OpenCV-Python is a library of Python bindings designed to solve computer vision problems. Python can be easily extended with C/C++, which allows us to write computationally intensive code in C/C++ and create Python wrappers that can be used as Python modules. OpenCV-Python is a Python wrapper for the original OpenCV C++ implementation. It makes use of Numpy.\footnote{\url{https://docs.opencv.org/4.x/index.html}}.

\paragraph*{\texttt{Robot Operating System}} is an open-source robotics middleware suite. It provides high-level hardware abstraction layer for sensors and actuators, an extensive set of standardized message types and services, and package management.\footnote{\url{https://www.ros.org/}}.

\paragraph*{\texttt{Pandas}} is an open source library providing high-performance, easy-to-use data structures and data analysis tools\footnote{\url{https://pandas.pydata.org}}.

\paragraph*{\texttt{Matplotlib}} is a comprehensive package for creating static, animated, and interactive visualisations in \texttt{Python}\footnote{\url{https://matplotlib.org}}.

\paragraph*{\texttt{Seaborn}} Seaborn \texttt{Python} is a data visualization library based on Matplotlib. It provides a high-level interface for drawing attractive statistical graphics. Because seaborn python is built on top of Matplotlib, the graphics can be further tweaked using Matplotlib tools and rendered with any of the Matplotlib backends to generate publication-quality figures.\footnote{\url{http://seaborn.pydata.org/}}.

\paragraph*{\texttt{scikit-learn}} is an open source package that provides simple and efficient tools for predictive data analysis, built on \texttt{NumPy}, \texttt{Scipy}, and \texttt{Matplotlib}\footnote{\url{https://scikit-learn.org}}.
\chapter{Methodologies}
\label{chap:methods}
\glsreset{rl}
\glsreset{il}
\chapter{Evaluation}
\label{chap:experiments}
\chapter*{Conclusion and Perspectives}
\addcontentsline{toc}{chapter}{Conclusion and Perspectives}
\label{chap:concl}
This chapter presents first, in \ref{sec:concl}, concluding thoughts and finally suggestions for possible future research lines in \ref{sec:fut}.

\section{Concluding Thoughts}
\label{sec:concl}
%Il vento è un problema attualmente. Il test è stato eseguito con un drone di fascia media/bassa, non è stabile abbastanza e quindi non si riesce a individuare la traiettoria come si deve. Per il tello nello spcifico, al di sotto dei 5km/h si riesce a lavorare, al di sopra, soprattutto sopra 10km/h diventa praticamente impossibile. Sarebbe quindi meglio provare su un drone più stabile, più resistente al vento. 

Our work makes a further step in the direction of Motion Tracking approaches, and aims to find feasible solutions in a practical scenario, that of drone filming. The hand gesture \gls{3d} tracking technology provides a detailed view of digitizing the measurement data, positional movement (tracked on the X, Y and Z axes of a \gls{3d} coordinate system) and orientation data calculated through Rotation (roll, pitch and yaw). This movement is reported in relation to the "detect" gesture. \\

\noindent \gls{3d} tracking technology bridges the gap between static images and dynamic movements and brings the physical world into digital interfaces (and vice-versa). This system is tied to its accuracy. In fact, the difference between arriving exactly at the destination or being off by miles (or millimetres in surgical navigation applications) is a requirement to keep absolutely in mind. Our system tries to establish itself not in an environment of absolute precision, but instead it tries to capture the user’s idea by helping him, for the success of the shooting. In this meaning, the original goal is reached with success. \\

\noindent The methods applied have been suitable and have led to the construction of a useful and working system. The main difficulties were those related to the estimation of orientation to obtain information on three-dimensionality, because it was necessary to reason on unconventional approaches. \\

\noindent The Tello is a mini drone that, despite containing several advanced features is mainly dedicated to novice users. Weighing only $80gr$, it does not remain stable in case of a very wind day. During the test phase, it was verified that trying to capture a trajectory directly from the drone’s camera in a day with a wind travelling over $10km/h$ is unfeasible. The drone is constantly pushed by the wind, making it unstable and generating too much noise on the trajectory not allowing the correct reading. Otherwise, with a wind slower than $5km/h$ the tests were carried out without any problem. This leads to the conclusion that it would be ideal to use a heavier drone, or at least a more wind resistant one, so that the latter can generate not too much noise to overpower the acquisition process of the \gls{3d} trajectory.

\section{Future Works}
\label{sec:fut}

%Potrebbe essere anche utile applicare un algortimo di stabilizzazione dell'immagine in tempo reale, così da avere un livello di precisione più alta nell'individuare la traiettoria. Da tenere conto anche un invio dei dati più veloce così da permettergli di fare una curva, oppure computarla prima così da evitare che se perso un pacchetto udp il drone compie altri tipi di traiettorie
Our contribution gives space to numerous developments: it might be interesting to think of a gestures combinations system, thus establishing a real language. Obviously, this would make user-drone interaction more complex. Even though benefits are more interesting types of applications as for example combinations of parameterizable programmed actions. In fact, if two different gestures with the same hand (at different times) are done, then two actions are performed whose intensity could be determined by the speed of passage from one gesture to another or by the direction in which this passage is made. \\

\noindent Since the orientation estimation is performed, it can be also exploited in such a way that this information is effectively used to make shots. This would permit the drone not only to respect the position in space, but also the orientation. In the case of Tello, is it possible handle only the yaw. However, there are plenty robots (like mechanical arms) capable to control also the row and pitch. Mediapipe gives a way to locate more than one hand in the scene. This means that it may be possible to combine not only different gestures with the same hand as time goes further, but also multiple hands with different gestures in the same instant or as time goes further.
\endgroup

\appendix %optional, use only if you have an appendix

\begingroup
\chapter{List of Acronyms}
\label{chp:acronyms}
\let\clearpage\relax
%\glsaddall  % comment to hide unused acronyms
\vspace*{-120pt}
\printglossary[type=acronym]
\endgroup

\backmatter

\nocite{*}
\bibliography{biblio}
%\cleardoublepage
%\theindex %optional, use only if you have an index, must use
%\makeindex in the preamble

\end{document}
